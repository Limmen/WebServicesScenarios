\documentclass[a4paper, 11pt]{article}
\usepackage{amsmath}
\usepackage{amssymb}
\usepackage[T1]{fontenc}
\usepackage[utf8x]{inputenc}
\usepackage{lmodern}
\usepackage{graphicx}
\graphicspath{ {./images/} }
\usepackage[english]{babel} 
\usepackage{natbib}
\usepackage{cite}
\usepackage[parfill]{parskip}
\usepackage{enumerate}
\usepackage{float}%for image positions
\usepackage{hyperref}
\hypersetup{
    colorlinks,
    citecolor=black,
    filecolor=black,
    linkcolor=black,
    urlcolor=black
}
\usepackage{amsthm}
\newtheorem{theorem}{Theorem}[section]
\newtheorem{lemma}[theorem]{Lemma}
\newtheorem{proposition}[theorem]{Proposition}
\newtheorem{axiom}[theorem]{Axiom}
\newtheorem{invariant}[theorem]{Invariant}
\newtheorem{breakpoint}[theorem]{Breakpoint}
\newtheorem{problem}{Problem}
\newtheorem{definition}{Definition} 
\usepackage{algorithm}
\usepackage{algpseudocode}
\usepackage{pifont}
\usepackage{multirow,array}
\usepackage{centernot}
\usepackage{comment} % enables the use of multi-line comments (\ifx \fi) 
\usepackage{lipsum} %This package just generates Lorem Ipsum filler text. 
\usepackage{fullpage} % changes the margin

\begin{document}
\noindent
\large\textbf{Homework 1} \hfill \textbf{Kim Hammar} \\
\normalsize ID2208 \hfill Due Date: 31 January 2017 \\
Programming Web-Services \hfill \\

\section*{Problem Statement}
The given task was to simulate an Employment Service Company that creates application profiles for job seekers. The profiles are XML-based and are produced by extracting data from multiple other sources, including Universities, Emplyment Officies, Online Company Service, and Employment Service Companies.

The task comprise designing five different XML schemas from different business domains, producing sample documents and performing XML processing. For practice-purpose the XML processing was required to use four different techniques: Document Object Model (DOM)-parsing, Simple API for XML (SAX)-parsing, Java Architecture for XML Binding (JAXB)-parsing, XQuery for extracting information,  and Extensible Stylesheet Language Transoformations (XSLT) for transforming XML documents. 
\section*{Main problems and solutions}

\section*{Conclusions}

\section*{Attachments}

\bibliography{references}{}
\bibliographystyle{plain}
\end{document}